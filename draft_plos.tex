% Template for PLoS
% Version 3.5 March 2018
%
% % % % % % % % % % % % % % % % % % % % % %
%
% -- IMPORTANT NOTE
%
% This template contains comments intended 
% to minimize problems and delays during our production 
% process. Please follow the template instructions
% whenever possible.
%
% % % % % % % % % % % % % % % % % % % % % % % 
%
% Once your paper is accepted for publication, 
% PLEASE REMOVE ALL TRACKED CHANGES in this file 
% and leave only the final text of your manuscript. 
% PLOS recommends the use of latexdiff to track changes during review, as this will help to maintain a clean tex file.
% Visit https://www.ctan.org/pkg/latexdiff?lang=en for info or contact us at latex@plos.org.
%
%
% There are no restrictions on package us`e within the LaTeX files except that 
% no packages listed in the template may be deleted.
%
% Please do not include colors or graphics in the text.
%
% The manuscript LaTeX source should be contained within a single file (do not use \input, \externaldocument, or similar commands).
%
% % % % % % % % % % % % % % % % % % % % % % %
%
% -- FIGURES AND TABLES
%
% Please include tables/figure captions directly after the paragraph where they are first cited in the text.
%
% DO NOT INCLUDE GRAPHICS IN YOUR MANUSCRIPT
% - Figures should be uploaded separately from your manuscript file. 
% - Figures generated using LaTeX should be extracted and removed from the PDF before submission. 
% - Figures containing multiple panels/subfigures must be combined into one image file before submission.
% For figure citations, please use "Fig" instead of "Figure".
% See http://journals.plos.org/plosone/s/figures for PLOS figure guidelines.
%
% Tables should be cell-based and may not contain:
% - spacing/line breaks within cells to alter layout or alignment
% - do not nest tabular environments (no tabular environments within tabular environments)
% - no graphics or colored text (cell background color/shading OK)
% See http://journals.plos.org/plosone/s/tables for table guidelines.
%
% For tables that exceed the width of the text column, use the adjustwidth environment as illustrated in the example table in text below.
%
% % % % % % % % % % % % % % % % % % % % % % % %
%
% -- EQUATIONS, MATH SYMBOLS, SUBSCRIPTS, AND SUPERSCRIPTS
%
% IMPORTANT
% Below are a few tips to help format your equations and other special characters according to our specifications. For more tips to help reduce the possibility of formatting errors during conversion, please see our LaTeX guidelines at http://journals.plos.org/plosone/s/latex
%
% For inline equations, please be sure to include all portions of an equation in the math environment.  For example, x$^2$ is incorrect; this should be formatted as $x^2$ (or $\mathrm{x}^2$ if the romanized font is desired).
%
% Do not include text that is not math in the math environment. For example, CO2 should be written as CO\textsubscript{2} instead of CO$_2$.
%
% Please add line breaks to long display equations when possible in order to fit size of the column. 
%
% For inline equations, please do not include punctuation (commas, etc) within the math environment unless this is part of the equation.
%
% When adding superscript or subscripts outside of brackets/braces, please group using {}.  For example, change "[U(D,E,\gamma)]^2" to "{[U(D,E,\gamma)]}^2". 
%
% Do not use \cal for caligraphic font.  Instead, use \mathcal{}
%
% % % % % % % % % % % % % % % % % % % % % % % % 
%
% Please contact latex@plos.org with any questions.
%
% % % % % % % % % % % % % % % % % % % % % % % %

\documentclass[10pt,letterpaper]{article}
\usepackage[top=0.85in,left=2.75in,footskip=0.75in]{geometry}

% amsmath and amssymb packages, useful for mathematical formulas and symbols
\usepackage{amsmath,amssymb}

% Use adjustwidth environment to exceed column width (see example table in text)
\usepackage{changepage}

% Use Unicode characters when possible
\usepackage[utf8x]{inputenc}

% textcomp package and marvosym package for additional characters
\usepackage{textcomp,marvosym}

% cite package, to clean up citations in the main text. Do not remove.
\usepackage{cite}

% Use nameref to cite supporting information files (see Supporting Information section for more info)
\usepackage{nameref,hyperref}

% line numbers
\usepackage[right]{lineno}

% ligatures disabled
\usepackage{microtype}
\DisableLigatures[f]{encoding = *, family = * }

% color can be used to apply background shading to table cells only
\usepackage[table]{xcolor}

% array package and thick rules for tables
\usepackage{array}

% algorithm table
\usepackage{algorithm,algorithmic}

% create "+" rule type for thick vertical lines
\newcolumntype{+}{!{\vrule width 2pt}}

% create \thickcline for thick horizontal lines of variable length
\newlength\savedwidth
\newcommand\thickcline[1]{%
\noalign{\global\savedwidth\arrayrulewidth\global\arrayrulewidth 2pt}%
\cline{#1}%
\noalign{\vskip\arrayrulewidth}%
\noalign{\global\arrayrulewidth\savedwidth}%
}

% \thickhline command for thick horizontal lines that span the table
\newcommand\thickhline{\noalign{\global\savedwidth\arrayrulewidth\global\arrayrulewidth 2pt}%
\hline
\noalign{\global\arrayrulewidth\savedwidth}}


% Remove comment for double spacing
%\usepackage{setspace} 
%\doublespacing

% Text layout
\raggedright
\setlength{\parindent}{0.5cm}
\textwidth 5.25in 
\textheight 8.75in

% Bold the 'Figure #' in the caption and separate it from the title/caption with a period
% Captions will be left justified
\usepackage[aboveskip=1pt,labelfont=bf,labelsep=period,justification=raggedright,singlelinecheck=off]{caption}
\renewcommand{\figurename}{Fig}

% Use the PLoS provided BiBTeX style
\bibliographystyle{plos2015}

% Remove brackets from numbering in List of References
\makeatletter
\renewcommand{\@biblabel}[1]{\quad#1.}
\makeatother



% Header and Footer with logo
\usepackage{lastpage,fancyhdr,graphicx}
\usepackage{epstopdf}
%\pagestyle{myheadings}
\pagestyle{fancy}
\fancyhf{}
%\setlength{\headheight}{27.023pt}
%\lhead{\includegraphics[width=2.0in]{PLOS-submission.eps}}
\rfoot{\thepage/\pageref{LastPage}}
\renewcommand{\headrulewidth}{0pt}
\renewcommand{\footrule}{\hrule height 2pt \vspace{2mm}}
\fancyheadoffset[L]{2.25in}
\fancyfootoffset[L]{2.25in}
\lfoot{\today}

%% Include all macros below

\newcommand{\lorem}{{\bf LOREM}}
\newcommand{\ipsum}{{\bf IPSUM}}

%% END MACROS SECTION


\begin{document}
\vspace*{0.2in}

% Title must be 250 characters or less.
\begin{flushleft}
{\Large
\textbf\newline{Simulation modeling of glycosylation effects on potassium channels of mouse cardiomyocytes} % Please use "sentence case" for title and headings (capitalize only the first word in a title (or heading), the first word in a subtitle (or subheading), and any proper nouns).
}
\newline
% Insert author names, affiliations and corresponding author email (do not include titles, positions, or degrees).
\\
Haedong Kim \textsuperscript{1},
Hui Yang \textsuperscript{1*},
Andrew R. Ednie \textsuperscript{2},
Eric S. Bennett \textsuperscript{2}
\\
\bigskip
\textbf{1} Complex Systems Monitoring, Modeling, and Control Laboratory, The Pennsylvania State University, University Park, Pennsylvania, United States of America
\\
\textbf{2} Department of Neuroscience, Cell Biology and Physiology, Wright State University, Dayton, Ohio, United States of America
\\
\bigskip

% Insert additional author notes using the symbols described below. Insert symbol callouts after author names as necessary.
% 
% Remove or comment out the author notes below if they aren't used.
%
% Primary Equal Contribution Note
% \Yinyang These authors contributed equally to this work.

% Additional Equal Contribution Note
% Also use this double-dagger symbol for special authorship notes, such as senior authorship.
% \ddag These authors also contributed equally to this work.

% Current address notes
% \textcurrency Current Address: Dept/Program/Center, Institution Name, City, State, Country % change symbol to "\textcurrency a" if more than one current address note
% \textcurrency b Insert second current address 
% \textcurrency c Insert third current address

% Deceased author note
% \dag Deceased

% Group/Consortium Author Note
% \textpilcrow Membership list can be found in the Acknowledgments section.

% Use the asterisk to denote corresponding authorship and provide email address in note below.
* huiyang@psu.edu

\end{flushleft}
% Please keep the abstract below 300 words
\section*{Abstract}
Dilated cardiomyopathy (DCM) is the third most common cause of heart failure and the primary reason for heart transplantation; upward of 70\% of DCM cases are considered idiopathic. Our in-vitro experiments showed that reduced hybrid/complex N-glycosylation in mouse cardiomyocytes is linked with DCM. Further, we observed direct effects of reduced N-glycosylation on K\textsubscript{v} gating. However, it is difficult to rigorously determine the effects of glycosylation on K\textsubscript{v} activity, because there are multiple K\textsubscript{v} isoforms in cardiomyocytes. Due to complex functions of K\textsubscript{v} isoforms, only the sum of K\textsuperscript{+} currents (I\textsubscript{Ksum}) can be recorded experimentally and decomposed later using exponential fitting to estimate component currents such as I\textsubscript{Kto}, I\textsubscript{Kslow}, I\textsubscript{Kss}. However, such estimation cannot adequately describe glycosylation effects and K\textsubscript{v} mechanisms. Here, we propose to develop simulation models of K\textsubscript{v} kinetics in mouse ventricular myocytes and calibrate their parameters using the in-vitro data under normal and reduced glycosylation conditions through ablation of the Mgat1 gene (i.e., Mgat1KO). Calibrated models facilitate the prediction of K\textsubscript{v} characteristics at different voltages that are not directly observed in the in-vitro experiments. A novel calibration procedure is also proposed to tackle nonlinear and high-dimensionality challenges in simulation models. Experimental results show that, in the Mgat1KO group, both I\textsubscript{Kto} and I\textsubscript{Kslow} densities are shown to be significantly reduced and the rate of I\textsubscript{Kslow} inactivation is much slower. The proposed approach has strong potential to couple simulation models with experimental data for gaining a better understanding of glycosylation effects on K\textsubscript{v} kinetics.

% Please keep the Author Summary between 150 and 200 words
% Use first person. PLOS ONE authors please skip this step. 
% Author Summary not valid for PLOS ONE submissions.   
% \section*{Author summary}
% Lorem ipsum dolor sit amet, consectetur adipiscing elit. Curabitur eget porta erat. Morbi consectetur est vel gravida pretium. Suspendisse ut dui eu ante cursus gravida non sed sem. Nullam sapien tellus, commodo id velit id, eleifend volutpat quam. Phasellus mauris velit, dapibus finibus elementum vel, pulvinar non tellus. Nunc pellentesque pretium diam, quis maximus dolor faucibus id. Nunc convallis sodales ante, ut ullamcorper est egestas vitae. Nam sit amet enim ultrices, ultrices elit pulvinar, volutpat risus.

\linenumbers

% Use "Eq" instead of "Equation" for equation citations.
\section*{Introduction}
% === I. INTRODUCTION ===================================================
% =======================================================================
Heart disease is the leading cause of death globally, accounting for 23\% of deaths in the U.S. in 2017 \cite{cdc2019deaths}. Dilated cardiomyopathy (DCM) is the third most common cause of heart failure and the most frequent reason for heart transplantation \cite{weintraub2017dilated}. DCM is characterized by enlarged and weakened ventricular chambers, and it is associated with systolic and contractile dysfunction that has a high risk to heart failure, with approximately 70\% of DCM cases regarded as idiopathic \cite{weintraub2017dilated, lakdawala2013dilated, hershberger2011update}. There has been consistent and increasing evidence of a link between aberrant glycosylation and heart failure \cite{gehrmann2003cardiomyopathy, footitt2009cardiomyopathy, marques2017cardiac}. Recently, we showed that reduction of hybrid/complex N-glycosylation in mouse cardiomyocytes, through ablation of the Mgat1\footnote{Mannosyl ($\alpha$-1,3-)-Glycoprotein $\beta$-1,2-N-acetylglucosaminyltransferase} gene that encodes a critical glycosyltransferase (GlcNAcT1\footnote{UDP-GlcNAc:$\alpha$-3-D-mannoside-$\beta$1,2-N-acetylglucosaminyltransferase}) (Mgat1KO model), is sufficient to cause DCM \cite{ednie2019reduced, ednie2019reduced2}. Mgat1KO mice develop DCM, heart failure, and 100\% die early, likely from ventricular arrhythmias resulting in sudden cardiac death. Further, Mgat1KO ventricular myocytes demonstrated altered electromechanical functions, including excitation-contraction (EC) coupling, are consistent with observed changes in electrical signaling caused by acute and downstream (disease-related) effects on voltage-gated ion channel (VGIC) gating and activity \cite{ednie2019reduced}. 

Thus, we investigated the impact of reduced hybrid/complex N-glycosylation through the lens of electrophysiology. Electrophysiology refers to studies of electrical properties in scales from ion channels in cells to organ systems. Electrical signaling in the heart has vital functions related to intra and extracellular communication, rhythmicity of heartbeats, and provides a driving force for contraction \cite{koenig2011voltage}. The action potential (AP) is the net transmembrane potential varying over time which results from the composite activities of different ion channels \cite{grant2009cardiac}. Fig~\ref{fig1}-(a) shows the simulated AP of a mouse ventricular apex myocyte using models adopted from \cite{bondarenko2004computer}. The AP is the result of orchestrated activities of various ionic currents described in Fig-(b). Even small changes in VGIC function can contribute to aberrant AP waveform and/or conduction, leading to arrhythmias. As illustrated in Fig~\ref{fig2}, Na\textsuperscript{+} channels (Na\textsubscript{v}) are responsible for AP initiation, or the depolarization phase, Ca\textsuperscript{2+} channels (Ca\textsubscript{v}) drive the prolonged depolarization phase particularly in larger species (“plateau”), while several K\textsuperscript{+} channel (K\textsubscript{v}) isoforms are collectively responsible for AP deactivation, i.e., repolarization. Like most transmembrane proteins, VGICs are heavily glycosylated membrane proteins \cite{ednie2011modulation}. Studies have proven that glycosylation can affect VGIC function predominantly. For example, we reported that a saturating, electrostatic effect of negatively charged sialic acids, which are typically attached to the terminal of glycan branches, significantly altered cardiac electrical signaling \cite{ednie2013expression, ednie2015reduced}. Our recent studies observed aberrant electrical signaling, e.g., prolonged APs and abnormal early re-activations, in Mgat1KO ventricular apex myocytes \cite{ednie2019reduced, ednie2019reduced2}. These observations strongly support direct and disease-related effects of reduced N-glycosylation on VGIC gating and activity.

\begin{figure}[!ht]
    \centering
    \caption{{\bf Simulation model of mouse myocytes.} 
    Simulation results of the (a) action potential and (b) relevant ionic currents (i.e., I\textsubscript{Kto}, I\textsubscript{Kslow}, I\textsubscript{Kss}, I\textsubscript{Na}, and I\textsubscript{CaL}) of mouse ventricular myocytes. Potassium currents have positive currents, while sodium and calcium currents are negative.}
    \label{fig1}
\end{figure}

\begin{figure}[!ht]
    \centering
    \caption{{\bf Human ventricular action potential.} 
    Predominant ion channels in human ventricular myocytes and their functional role in shaping an AP.}
    \label{fig2}
\end{figure}

Although these in-vitro experiments showed changes in activities of ion channels, there are certain limitations that include: 1) \textit{Detailed ion channel kinetics are difficult to determine rigorously using in-vitro experiments alone}. From the electrophysiological experiment, pathological electrical signaling can be observed through measurements of APs at the whole-cell level and currents at the ion-channel level. However, it is difficult to relate, rigorously, the impacts of VGIC gating changes to the altered AP waveform/conduction and vice versa. 2) \textit{Segregating different K\textsuperscript{+} currents (I\textsubscript{K}) is difficult using whole-cell recording experimental techniques only}. I\textsubscript{Ksum} is the result of joint activity of multiple isoforms with each isoform producing a different, but slightly overlapping (in activation and inactivation voltages) component of the total K\textsuperscript{+} current (I\textsubscript{Ksum}) that contributes to various portions of AP repolarization \cite{du2017, jerng2021light}. However, in-vitro experiments can only measure the sum of these component currents (I\textsubscript{Ksum}) using whole-cell voltage-clamp protocols. Even pharmacologic separation of K\textsubscript{v} current types is difficult, as the specificity of drugs for a single current type is not ideal. Hence, component K\textsuperscript{+} currents are usually estimated through multiple exponential fits of the decaying portion (due to channel inactivation) of the total K\textsuperscript{+} current \cite{brunet2004heterogeneous}. Although the exponential function illustrates the shape of a component-current trace well, it cannot adequately describe kinetic dynamics of the current, nor fully and rigorously distinguish among currents produced by different K\textsubscript{v} isoforms because of their slightly overlapping voltage-dependence of gating. As K\textsubscript{v} have a critical function in repolarizing the cell, it is imperative to model and examine K\textsubscript{v} isoforms thoroughly.

Therefore, this paper presents simulation modeling of dominant K\textsuperscript{+} currents in mouse ventricular myocytes and further calibrate their parameters using the in-vitro data under normal and reduced glycosylation conditions through ablation of the Mgat1 gene. Specifically, we propose an optimization procedure to calibrate simulation models of potassium channel isoform activity, I\textsubscript{Kto} and I\textsubscript{Kslow}, in mouse cardiomyocytes.  Fig~\ref{fig3} shows  major component K\textsuperscript{+} currents, including I\textsubscript{Ksum}, I\textsubscript{Kto}, I\textsubscript{Kslow} and I\textsubscript{Kss}. Key statistics of component currents, amplitudes, and inactivation rates ($\tau$'s) are measured from in-vitro experimental data with a 4.5 s voltage-clamp protocol (see details in In-vitro Experimental Data Section in Methods). The major contributions of this paper are summarized as follows.
\begin{itemize}
    \item Simulation models are integrated with in-vitro experiments to investigate the effects of reduced N-glycosylation on K\textsubscript{v} activity of mouse myocytes. Computer simulation helps demonstrate the voltage-gating mechanism and conductance kinetics that cannot be readily available through traditional exponential fitting. 
    \item A self-breeding Genetic Algorithm procedure is developed to calibrate simulation models of voltage-gated K\textsuperscript{+} channel of mouse ventricular apex myocytes based on key characterisitcs and raw data obtained from in-vitro experiments.
    \item Calibrated models facilitate the prediction of K\textsubscript{v} characteristics at different voltages that are not observed in the in-vitro experiments. In the Mgat1KO group, both I\textsubscript{Kto} and I\textsubscript{Kslow} densities are shown to be significantly reduced and the rate of I\textsubscript{Kslow} inactivation is much slower.
    \item Simulation modeling of cardiac myocytes is conducive to gain a better understanding of detailed K\textsubscript{v} kinetics, as well as how reduced glycosylation through ablation of Mgat1 gene impacts K\textsubscript{v} kinetics.
\end{itemize}

\begin{figure}[!ht]
    \centering
    \caption{{\bf Dominant K\textsuperscript{+} cureents.} 
    Example of K\textsuperscript{+} current decomposition into three component currents. Each component K\textsuperscript{+} current has different characteristics in terms of their peak and decaying phase.}
    \label{fig3}
\end{figure}

\section*{Background}
% === II. RESEARCH BACKGROUND ===========================================
% =======================================================================
\subsection*{Glycosylation and DCM}
Protein glycosylation is an essential cellular process that impacts many cell functions \cite{marques2017cardiac}. Briefly, protein glycosylation is the sequential co-/post-translational process of attaching sugar residues (glycans) to proteins. Cardiac VGICs are heavily glycosylated proteins with upwards of 30\% of their mass consisting of N- and O-linked glycans. A growing number of cardiac diseases, including DCM and hypertrophic cardiomyopathy, can present with concurrent, albeit, modest changes in glycosylation \cite{gehrmann2003cardiomyopathy, footitt2009cardiomyopathy, molina2013differential, marques2017cardiac}. Mgat1 expression was implicated in cardiac function and shown to be down-regulated in human end-stage idiopathic DCM. Genome-wide searches identified changes in glycosylation-related gene expression in human idiopathic DCM, including glycosyltransferases \cite{barrans2002global, hwang2002microarray, yung2004gene}; and proteomic/glycomic studies show changes in serum N-glycosylation in heart disease models and in humans with DCM risk factors \cite{nishio1995identification, knezevic2009variability, miura2016glycomics, nagai2016aberrant, yang2015glycoproteins}. Models of DCM/heart failure were associated with subtle changes in glycosylation of proteins involved in electromechanical processes, and ~20\% of patients with congenital disorders of glycosylation (CDG) present with cardiac deficits, including idiopathic DCM \cite{gehrmann2003cardiomyopathy, marques2017cardiac}. The data suggest a correlation between modest changes in extracellularly facing glycosylation and DCM/heart disease. 

\subsection*{Simulation Modeling of Mouse Ventricular Myocytes}
Electrophysiology was pioneered by Hodgkin and Huxley with a series of well-known experiments on the squid giant axon \cite{hodgkin1952quantitative}. Their simulation model was built upon two fundamental principles observed in the experimental data. 1) Cells are selectively activated by different ion channels. 2) Activities of ion channels to produce ionic currents are controlled through voltage changes, from which the name ``voltage-gated ion channels'' originates. The voltage-gating mechanism is encoded in the simulations model with partial differential equations. Not only are simulation models compatible with data and descriptive, but they also simulate changes of AP waveforms and ion channel activity \cite{mullins2013mathematical}. Thus, models provide connections among ionic currents, action potentials, and disease effects. However, significant domain knowledge and in-vitro investigations are required, because simulation models often have complex structures based on nonlinear functions and differential equations with a high-dimensional set of parameters.

In the literature, optimization procedures have been proposed to fit cardiomyocyte simulation models to electrophysiological experimental data. For example, Na\textsubscript{v} activity was coupled with in-vitro data of mouse cardiomyocyte using fractional factorial design and a Gaussian process model \cite{du2015statistical}. Fractional factorial designs were used to find significant control variables from the entire set of tuning parameters to reduce dimensionality. The Gaussian process model, which is computationally cheaper than the original Na\textsubscript{v} model, has served as a surrogate model to find optimal parameters. Characteristic curves such as steady-state activation (SSA), steady-state inactivation (SSI), and time constant of inactivation at different voltage were used to calculate goodness-of-fit as an objective function of optimization to minimize discrepancies with in-vitro experiment data. However, they are derived from current traces recorded during in-vitro experiments and may not always be available. For example, in our Mgat1KO data, currents were too small at early-activating voltages to reliably calculate the characteristic curves.  

\subsection*{K\textsubscript{v} Isoforms}
The orchestrated activity of different K\textsubscript{v} isoforms produces multiple component currents, as illustrated in Fig~\ref{fig1}-(b) and Fig~\ref{fig3} However, it is difficult to rigorously delineate the multiple component K\textsuperscript{+} currents through voltage-clamp experiments alone because only the sum of K\textsuperscript{+} currents is measured and recorded. Thus, I\textsubscript{Ksum} is usually decomposed mathematically by fitting the peak and decaying portion of each component current trace with an exponential function \cite{brunet2004heterogeneous}. As given in Eq~\ref{eq1}, a standard exponential function has three parameters: amplitude ($\mathrm{A}$), time constant ($\tau$), and constant offset value ($\mathrm{C}$). $t$ is time in millisecond. Amplitude refers to the value of the peak and $\tau$ to the time constant when current reduces by $e^{-1}$ (~63\%) of the peak. 

\begin{equation}
    \mathrm{I} = \mathrm{A}e^{-t/\tau} + \mathrm{C}
    \label{eq1}
\end{equation}

In mouse ventricular myocytes, I\textsubscript{Ksum} has often been decomposed into three dominant currents, with one current having a high peak and decaying rapidly (I\textsubscript{Kto}), a second current having a smaller peak and decaying slowly (I\textsubscript{Kslow}), and the third current type being constant (I\textsubscript{Kss}), so that a bi-exponential function with a constant component, as in Eq~\ref{eq2}, was used to separate different current components for 4.5-second protocols \cite{ednie2015reduced}. Fig~\ref{fig3} shows an example of the decomposition of I\textsubscript{Ksum} into three component currents. I\textsubscript{Kto} is a rapidly inactivating transient outward current that has a high peak at the very beginning of activation and rapid inactivation. I\textsubscript{Kslow}, a delayed rectifier-type current, has a low peak and longer inactivation phase. I\textsubscript{Kss}, a non-inactivating steady-state current, remains constant during the course of depolarization. In Eq~\ref{eq2}, $\mathrm{A}_{Kto}$ and $\mathrm{A}_{Kslow}$ are the amplitudes; $\tau_{Kto}$ and $\tau_{Kslow}$ are time constants of I\textsubscript{Kto} and I\textsubscript{Kslow}, respectively. $\mathrm{A}_{Kss}$ is the constant current I\textsubscript{Kss}. Although bi-exponential fitting captures essential characteristics of three major component K\textsuperscript{+} currents and describes the I\textsubscript{Ksum} waveform, it cannot provide detailed kinetic dynamics of K\textsubscript{v} isoforms \cite{plumlee2016calibrating}. 

\begin{equation}
    \mathrm{I}_{\mathrm{Ksum}} = \mathrm{A}_{Kto} e^{-t/\tau_{Kto}} + \mathrm{A}_{Kslow} e^{-t/\tau_{Kslow}} + \mathrm{A}_{Kss}
    \label{eq2}
\end{equation}

\section*{Methods}
% === III. METHODS ======================================================
% =======================================================================
In this study, we propose an optimization procedure to calibrate simulation models of potassium channel isoform activity in mouse cardiomyocytes. Coupling in-vitro data of K\textsubscript{v} with in-silico models has advantages that bi-exponential fitting cannot provide. For example, ionic currents, action potentials, and conductance can be determined more rigorously; and models allow for the simulation and prediction of glycosylation conditions that are not observed in the in-vitro experiment. 

Specifically, this paper presents a new method that fits the simulation models directly to amplitudes and $\tau$'s, which are the most critical characteristics to describe component K\textsuperscript{+} current traces. Eq~\ref{eq3} shows the objective function of the suggested optimization procedure, where $\hat{A_i}$ and $\hat{\tau_i}$ are estimations of amplitude and time constant from the simulation model. Potassium channels from mouse ventricular myocytes are modeled and parameterized based on partial differential equations. Then, full factorial designs are used to discover sensitive variables from the potential set of parameters. Further, a genetic algorithm-based heuristic optimization method is developed to calibrate the models to the data.

\begin{equation}
    \min (|\mathrm{A}_i - \hat{\mathrm{A}_i}| + |\tau_i - \hat{\tau_i}|), \ i \in \{\mathrm{Kto},\ \mathrm{Kslow}\}
    \label{eq3}
\end{equation}

\subsection*{In-vitro Experimental Data}
Recently, we reported electrophysiological experiment data to investigate the impact of reduced hybrid/complex N-glycosylation on left ventricular cardiomyocyte activity, through deletion of the Mgat1 gene, which encodes a critical glycosyltransferase (GlcNAcT1) \cite{ednie2019reduced, ednie2019reduced2}. The detailed process of creation of Mgat1KO (Mgat1 Knock Out) strain, features of the cardiomyocyte-specific Mgat1KO strain, breeding, genotyping, and selection of wild type (WT) animals were previously described in \cite{ednie2019reduced2}. In our previous study of K\textsubscript{v} activities in the Mgat1KO \cite{ednie2019reduced}, cells of 12 to 20-week-old mice were used and numbers of observation for WT and Mgat1KO are 35 and 38 respectively.

As mentioned earlier, I\textsubscript{Ksum} was measured using whole-cell voltage clamp protocols. Cells were held at -70 mV and then depolarized by 10 mV voltage steps from -50 mV to 50 mV for 4.5 seconds. Note that this paper does not include the 25s depolarization, which often helps distinguish I\textsubscript{Kslow1} (K\textsubscript{v}1.5 activity) and I\textsubscript{Kslow2} (K\textsubscript{v}2.1 activity). Our ongoing experiments focus on voltage-clamp experiments on 25-second periods. The resulted data will be further studied and calibrated with simulation models in the future work.

\begin{figure}[!ht]
    \centering
    \caption{{\bf In-vitro experimental results.} 
    In-vitro experimental results, adopted from \cite{ednie2019reduced}, of averages and error bars of (a) the peaks and constant current, and (b) inactivation time constant. As reported in \cite{ednie2019reduced}, significant differences between WT and Mgat1KO at $p \leq 0.05$ are indicated by an *. ($n=35$ for WT and $n=38$ for Mgat1KO).}
    \label{fig4}
\end{figure}

Fig~\ref{fig4} shows our in-vitro experimental data about key statistics on I\textsubscript{Kto} and I\textsubscript{Kslow}. Peaks of both currents are reduced in the Mgat1KO group. Steady-state current I\textsubscript{Kss}, remaining constant during the depolarization, is also reduced in the Mgat1KO group. To be specific, $\mathrm{A}_{Kslow}$ is most reduced compared to WT, by ~77\%. In addition, $\tau_{Kslow}$ of the Mgat1KO group is ~30\% larger than the WT group, which means I\textsubscript{Kslow} is prolonged with reduced N-glycosylation. However, $\tau_{Kto}$ does not show significant difference.

\subsection*{Simulation Models of I\textsubscript{Kto}, and I\textsubscript{Kslow}}
In this paper, cell AP is modeled as an orchestrated activity of various ion channels and electrogenic transport proteins, as given in Eq~\ref{eq4}. In this equation, $C_m$ is the membrane capacitance, and there are several ionic currents: L-type calcium current (I\textsubscript{CaL}), calcium pump current (I\textsubscript{p(Ca)}), Na\textsuperscript{+}/Ca\textsuperscript{2+} exchange current (I\textsubscript{NaCa}), calcium background current (I\textsubscript{Cab}), fast Na\textsuperscript{+} current (I\textsubscript{Na}), background Na\textsuperscript{+} current (I\textsubscript{Nab}), Na\textsuperscript{+}/K\textsuperscript{+} pump current (I\textsubscript{NaK}), fast transient outward K\textsuperscript{+} current (I\textsubscript{Kto,f}), slower transient outward K\textsuperscript{+} current (I\textsubscript{Kto,s}), which is essentially missing in apex myocytes), time independent K\textsuperscript{+} current (I\textsubscript{K1}), slow delayed rectifier K\textsuperscript{+} current (I\textsubscript{Ks}), ultrarapidly activating delayed rectifier K\textsuperscript{+} current (I\textsubscript{Kur}), non-inactivating steady-state K\textsuperscript{+} current (I\textsubscript{Kss}), rapidly delayed rectifier K\textsuperscript{+} current (I\textsubscript{Kr}), Ca\textsuperscript{2+}-activated Cl\textsuperscript{-} current (I\textsubscript{Cl,Ca}), and stimulus current (I\textsubscript{stim}). Although there are seven currents pertinent to K\textsubscript{v} dynamics in this model, as mentioned earlier, three current types, I\textsubscript{Kto}, I\textsubscript{Kslow}, and I\textsubscript{Kss}, are dominant K\textsuperscript{+} currents in a ventricular apex myocyte \cite{nerbonne2005molecular} (also see Fig~\ref{fig4}). In this study, we modeled and calibrated I\textsubscript{Kto} and I\textsubscript{Kslow}.

\begin{equation}
    \begin{split}
    -C_{m}\frac{dV}{dt} &= \mathrm{I}_{\mathrm{CaL}}+\mathrm{I}_{\mathrm{p}(Ca)}+\mathrm{I}_{\mathrm{NaCa}}+\mathrm{I}_{\mathrm{Cab}}\\
    &+\mathrm{I}_{\mathrm{Na}}+\mathrm{I}_{\mathrm{Nab}}+\mathrm{I}_{\mathrm{NaK}}+\mathrm{I}_{\mathrm{Kto},f}\\
    &+\mathrm{I}_{\mathrm{Kto},s}+\mathrm{I}_{\mathrm{K1}}+\mathrm{I}_{\mathrm{Ks}}+\mathrm{I}_{\mathrm{Kur}}\\
    &+\mathrm{I}_{\mathrm{Kss}}+\mathrm{I}_{\mathrm{Kr}}+\mathrm{I}_{\mathrm{Cl},Ca}+\mathrm{I}_{\mathrm{stim}}
    \end{split}
    \label{eq4}
\end{equation}

There are two major modeling schemes for VGICs, Markov models and Hodgkin-Huxley models. Hodgkin-Huxley modeling is commonly used to formulate K\textsuperscript{+} currents for various species, for example, not only for mouse ventricular myocytes but also for rabbit and human ventricular myocytes. In this study, I\textsubscript{Kto} and I\textsubscript{Kslow} are designed using Hodgkin-Huxley modeling as given in Eq~\ref{eq5} and Eq~\ref{eq6} respectively. Hodgkin-Huxley models have two state variables for describing activation and inactivation. For example, in Equation (5), $a_{Kto}$ and $i_{Kto}$ are two state variables of activation and inactivation, respectively. $G_{Kto}$ is the maximum conductance of K\textsubscript{v}4.2, and $(V-E_k)$ is the difference of voltage $V$ and the K\textsuperscript{+} Nernst potential $E_k$. Similarly, the I\textsubscript{Kslow} model is given as Equation (6), which also include two state variables for activation and inactivation gates. Relevant parametric equations of transition rates and time constants for the gating mechanisms are given below. Maximum conductance variables $G_{Kto}$ and $G_{Kslow}$ are parameterized as variables to be calibrated during the optimization; thereby, conductance values for different cells are estimated.

\begin{equation}
    \mathrm{I}_{Kto} = G_{Kto}a_{Kto}^{3}i_{Kto}(V-E_K)
    \label{eq5}
\end{equation}
\begin{align*}
    &\frac{da_{Kto}}{dt} = \alpha_{a}(1-a_{Kto}) - \beta_{a}a_{Kto} \\
    &\frac{di_{Kto}}{dt} = \alpha_i(1-i_{Kto}) - \beta_i i_{Kto} \\
    &\alpha_a = 0.18064 e^{0.03577(V+x_1)} \\
    &\beta_a = 0.395 e^{-0.06237(V+x_2)} \\
    &\alpha_i = \frac{0.000152 e^{-(V+x_3)/x_4}}{0.067083 e^{-(V + x_5)/x_4} + 1} \\
    &\beta_i = \frac{0.00095 e^{(V+x_6)/x_7}}{0.051335 e^{(V+x_6)/x_7} + 1} \\
\end{align*}

\begin{equation}
    \mathrm{I}_{Kslow} = G_{Kslow}a_{Kslow}i_{Kslow}(V-E_k)
    \label{eq6}
\end{equation}
\begin{align*}
    &\frac{a_{Kslow}}{dt} = \frac{a_{ss}-a_{Kslow}}{\tau_{a}} \\
    &\frac{i_{Kslow}}{dt} = \frac{i_{ss}-i_{Kslow}}{\tau_{i}} \\
    &a_{ss} = \frac{1}{1+e^{-(V+x_1)/x_2}} \\
    &i_{ss} = \frac{1}{1+e^{(V+x_3)/x_4}} \\
    &\tau_{a} = 0.493 e^{-0.0629V}+x_5 \\
    &\tau_{i} = x_6 - \frac{170}{1+e^{(V+x_7)/x_8}}
\end{align*}

\subsection*{Variable Screening}
In in-silico experiments, it is critical to identify a sparse subset of sensitive variables (or model parameters to be calibrated) to reduce computational burden and improve modeling accuracy. The curse of dimensionality causes the dramatic surge of required computing resources when the number of variables increases and counter-intuitive geometric properties, making the learning procedure difficult. While a sparse subset of variables restricts the model flexibility to fit data, the opposite also has a detrimental effect on modeling accuracy. We used two-level factorial designs to perform sensitivity analysis and screen the variables that impact the model outputs of our interest. Variables were adjusted at two different levels to assess their effects on model outputs and K\textsubscript{v} characteristics. Table~\ref{table1} shows the final subset of sensitive variables.

% Place tables after the first paragraph in which they are cited.
\begin{table}
    \centering
    \caption{\bf{Sensitive variables identified by 2-level factorial designs.}}
    \begin{tabular}{ll}
        \hline
        Model & Selected Variables  \\ 
        \hline
        I\textsubscript{Kto} & $x_1, x_2, x_3, x_6, x_7, G_{Kto}$ \\
        I\textsubscript{Kslow} & $x_1, x_2, x_3, x_4, x_6, G_{Kslow}$ \\
        \hline
    \end{tabular}
    \label{table1}
\end{table}

\subsection*{Optimization Process}
Furthermore, we proposed a novel metaheuristics optimization method based on the genetic algorithm (GA). Metaheuristics optimization is a higher-level procedure, which means that it is not problem-specific, but to find sufficiently satisfactory solutions for complex optimization problems efficiently. Most metaheuristics methods have common characteristics: they are inspired by nature and often involve local-search heuristic methods without gradient or Hessian matrix. GA mimics the evolution of genes based on natural selection, and its procedure is described in Algorithm~\ref{alg1}.
\begin{algorithm}[!ht]
    \caption{Standard Genetic Algorithm}
    \begin{algorithmic}[1]
        \renewcommand{\algorithmicrequire}{\textbf{Input:}}
        \renewcommand{\algorithmicensure}{\textbf{Output:}}
        \REQUIRE Stopping criterion, Population size $N$, Number of solutions to be selected $k$
        \ENSURE Best solution $\mathbf{x^*} \in \mathbb{R}^n$
        \\ \textit{Initialization}:
        \STATE Generate a random initial population of size $N$
        \\ \textit{LOOP Process}
        \WHILE{Satisfying stopping criterion}
        \STATE Fitness - Evaluate fitness of each solution in current population
        \STATE Selection - Select $k$ solutions with highest fitness and update best solution $\mathbf{x^{*}}$
        \STATE Breeding - Generate additional $N-k$ new solutions by doing crossover elements of top $k$ solutions randomly
        \STATE Mutation - Add random noise 
        \STATE Update current population with solutions generated through Step 4 and Step 6
        \ENDWHILE
        \RETURN $\mathbf{x^{*}}$ 
    \end{algorithmic}
    \label{alg1}
\end{algorithm}

In the standard GA, a new population is constructed from superior solutions of the current population and new solutions are reproduced in Breeding step by so-called crossover. As illustrated in Fig~\ref{fig5}, the crossover is a process of mixing up elements of superior solutions in the current population so that this process is also called reproduction. This procedure is expected to discover better candidates for combinatorial optimization problems by searching combinations of elite solutions. Although crossover is an intuitive searching method for combinatorial or discrete optimization problems, exploring combinations is not appropriate for continuous variables. Hence, in this paper, we presented a new method named \textit{self-breeding genetic algorithm} that reproduces new generations without crossover but directly breeds them from each superior solution. The procedure is described in Algorithm~\ref{alg2}, and Fig~\ref{fig5} illustrates the difference between standard GA and proposed \textit{self-breeding GA}.

\begin{figure}[!ht]
    \centering
    \caption{{\bf Method illustration.} 
    An illustration of proposed self-breeding genetic algorithm and comparison with standard genetic algorithm.}
    \label{fig5}
\end{figure}

\begin{algorithm}[!ht]
    \caption{Self-breeding Genetic Algorithm}
    \begin{algorithmic}[1-=0+-*]
        \renewcommand{\algorithmicrequire}{\textbf{Input:}}
        \renewcommand{\algorithmicensure}{\textbf{Output:}}
        \REQUIRE Tolerance $(\epsilon_{A}, \epsilon_{\tau})$, Target amplitude and time constant $(A, \tau)$, Population size $N$, Selection size $k$, Breeding size $s$, Lower bound $\mathbf{l} \in \mathbb{R}^p$, Upper bound $\mathbf{u} \in \mathbb{R}^p$
        \ENSURE Model discrepancy $(\delta_{A}, \delta_{\tau})$, Solution $\mathbf{x}^* \in \mathbb{R}^p$
        \\ \textit{Initialization}:
        \STATE Construct initial population by generating $N$ solutions from $Unif(\mathbf{l}, \mathbf{u})$
        \\ \textit{LOOP Process}
        \WHILE{$\delta_{A} \geq \epsilon_{A}$ and $\delta_{\tau} \geq \epsilon_{\tau}$}
        \STATE Fitness - Evaluate fitness of each solution in current population        
        \STATE Update $\delta_{A}$ and $\delta_{\tau}$
        \STATE Selection - Select $k$ solutions $\mathbf{x_i}$, $i=1,2,...,k$, with highest fitness and update best solution $\mathbf{x^{*}}$
        \STATE Calculate variance $\mathbf{\sigma}^2_{i}$ of top $k$ solutions and calculate pooled variance $\mathbf{\sigma}_{p}$ using recent $w$ recent variances, $\mathbf{\sigma}^2_{i}$, $\mathbf{\sigma}^2_{i-1}$, ..., $\mathbf{\sigma}^2_{i-w+1}$
        \FOR {$i = 1$ to $k$}
            \STATE Generate $s$ solutions of $\mathbf{x_i} + N(0,\mathbf{\sigma}^2)$
        \ENDFOR
        \STATE Update current population with top $k$ solutions generated in Step 4 and $ks$ solutions generated through Step 6 and Step 8
        \ENDWHILE
        \RETURN $(\delta_{A}, \delta_{\tau})$ and $\mathbf{x}^*$
    \end{algorithmic}
    \label{alg2}
\end{algorithm}

% Results and Discussion can be combined.
\section*{Results}
% === IV. RESULTS =======================================================
% =======================================================================
The self-breeding GA was applied to a pair of mean values of $\mathrm{A}_i$ and $\tau_i$ in Fig~\ref{fig4}, $i \in \{to,\ Kslow\}$, for each group, WT and Mgat1KO. This calibration process was repeated 30 times. The algorithm stopped when discrepancies of amplitude and time constant were less than tolerances. Tolerances for amplitude and time constant were (0.1, 1.0) for I\textsubscript{Kto} and (0.1, 5.0) for I\textsubscript{Kslow}, which are smaller than the standard error of mean values. Table~\ref{table2} shows means and standard errors of calibrated parameters found by the suggested optimization process. The simulation models and optimization process were implemented in MATLAB 2020a, and the codes are available at \url{https://github.com/haedong31/20_Kv_simulation}.

\begin{table}
    \centering
    \caption{\bf{Calibrated model parameters}}
    \begin{tabular}{lrrlrr}
        \hline
        \multicolumn{3}{l}{I\textsubscript{Kto}} & \multicolumn{3}{l}{I\textsubscript{Kslow}} \\ 
        \hline
        Parameter & WT & Mgat1KO & Parameter & WT & Mgat1KO \\
        \hline
        $x_1$ & $27.4 \pm 4.3$ & $33.9 \pm 4.0$ & $x_1$ & $33.2 \pm 10.3$ & $18.0 \pm 14.0$ \\
        $x_2$ & $42.6 \pm 4.8$ & $51.5 \pm 7.7$ & $x_2$ & $7.6 \pm 0.9$ & $10.7 \pm 1.0$ \\
        $x_3$ & $42.5 \pm 5.4$ & $48.4 \pm 5.2$ & $x_3$ & $54.3 \pm 9.0$ & $56.4 \pm 7.3$ \\
        $x_6$ & $55.7 \pm 2.4$ & $54.7 \pm 2.9$ & $x_4$ & $12.4 \pm 2.0$ & $10.6 \pm 1.3$ \\
        $x_7$ & $36.0 \pm 0.8$ & $36.0 \pm 1.0$ & $x_6$ & $1432.4 \pm 3.1$ & $1853.9 \pm 11.5$ \\
        $G_{Kto}$ & $0.193 \pm 0.002$ & $0.142 \pm 0.001$ & $G_{Kslow}$ & $0.153 \pm 0.001$ & $0.075 \pm 0.015$ \\
        \hline
    \end{tabular}
    \label{table2}
\end{table}

Fig~\ref{fig6} shows the current traces generated by the calibrated simulation models and exponential fitting with the mean amplitudes and time constants from in-vitro experiment data. Simulation models are shown to be compatible with results of exponential fitting in the electrophysiological data. The solid gray lines represent simulated current traces with 30 repetitions. Except for I\textsubscript{Kslow} of the Mgat1KO, it is difficult to distinguish the two traces through a visual inspection. For this one, there was a slight dissimilarity during the early decay phase. Also, before applying the clamp voltage, it showed instability that currents had small positive values under the holding potential. Due to the small peak but large time constant, the trace of the Mgat1KO is flattened compared to a standard exponential function, which sharply decreases after the peak. This change is expected to show modeling differences between WT and Mgat1KO.

\begin{figure}[!ht]
    \centering
    \caption{{\bf Comparison of simulation modeling and exponential fitting.} 
    The comparison of current traces between calibrated simulations models (gray line) and benchmark exponential fitting (blue lines: WT; red lines: Mgat1KO). Each plot includes 30 replications of simulations results.}
    \label{fig6}
\end{figure}

\subsection*{Simulation Comparison between WT and Mgat1KO}
To further investigate additional differences between the Mgat1KO and WT groups, current traces were simulated at voltages between -60 mV and 50 mV in 10-mV increments as shown in Fig~\ref{fig7}. Simulated current traces show different patterns as voltage changes between two groups. For I\textsubscript{Kslow} in Fig~\ref{fig7}, as voltage decreases, the amplitude declines rapidly, for Mgat1KO, while reducing evenly in WT. Because of the rapid reduction, I\textsubscript{Kslow} traces for Mgat1KO below 0 mV depolarizations are negligible. In addition, from the perspective of decay phase, the current decreases slowly for I\textsubscript{Kslow} in the Mgat1KO group. It is predicted that there are differences in conductance values between WT and Mgat1KO.

\begin{figure}[!ht]
    \centering
    \caption{{\bf Prediction of current traces.} 
    Predicted current traces of I\textsubscript{Kto} and I\textsubscript{Kslow} for each group. (a) Current traces of WT and (b) for Mgat1KO. Clamp voltages of 4.5 seconds were applied to from -60 mV to +50 mV by 10-mV increments from the holding potential -70 mV.}
    \label{fig7}
\end{figure}

Fig~\ref{fig8} further demonstrates the differences in predicted current density to voltage relationship between WT and Mgat1KO. For both I\textsubscript{Kto} and I\textsubscript{Kslow}, channels in Mgat1KO are less active at a given depolarized activation voltage compared to WT. The difference in the current peaks between the two groups is more prominent at depolarizations greater than -10 mV. This gap is more significant in I\textsubscript{Kslow} than I\textsubscript{Kto}. At all voltages, the Mgat1KO I\textsubscript{Kslow} is smaller than the WT I\textsubscript{Kslow}. This is consistent with in-vitro experimental results in which the average amplitude of I\textsubscript{Kto} and I\textsubscript{Kslow} were significantly reduced with N-glycosylation perturbation (in the Mgat1KO), and the reduction in Mgat1KO I\textsubscript{Kslow} was bigger than for Mgat1KO I\textsubscript{Kto}. 

\begin{figure}[!ht]
    \centering
    \caption{{\bf Prediction of density to voltage relationship.} 
    Predicted current density to voltage relationship for I\textsubscript{Kto} and I\textsubscript{Kslow} under Mgat1KO (red solid line with circle markers) and WT conditions (blue dotted line with square markers). Amplitudes were recorded for voltage steps ranging from -60 mV to 50 mV by 10-mV increments.} 
    \label{fig8}
\end{figure}

K\textsubscript{v} has unique inactivation kinetics that makes it difficult to dinstinguish component K\textsuperscript{+} currents. To investigate its inactivation gating kinetics, inactivation time constants at various voltages were simulated as shown in Fig~\ref{fig9}. $\tau_{Kto}$ does not show a significant difference between Mgat1KO and WT when the depolarization is greater than -20 mV. There are gaps in the case where voltages are less than -20 mV, but the uncertainty represented by the error bars are too wide to make reliable predictions. $\tau_{Kslow}$ shows consistent differences, with the Mgat1KO inactivating significantly more slowly than WT at all voltages. Fig~\ref{fig10} provides further information of the steady-state inactivation (SSI) rate. The SSI relationships are not significantly different between Mgat1KO and WT for either current type. These data, the longer transition times from open to inactivated state and the lack of a shift in the voltage dependence of K\textsubscript{v} distribution between open and inactivated states (SSI) for less glycosylated K\textsubscript{v} (Mgat1KO), are consistent with our previous studies \cite{schwetz2010n, ednie2015reduced, du2017}

\begin{figure}[!ht]
    \centering
    \caption{{\bf Prediction of inactivation time constant.} 
    The prediction of inactivation time constants for I\textsubscript{Kto} and I\textsubscript{Kslow} with voltage steps from -60 mV to 50 mV by 10 mV steps.} 
    \label{fig9}
\end{figure}

\begin{figure}[!ht]
    \centering
    \caption{{\bf Prediction of SSI.} The prediction of SSI for I\textsubscript{Kto} and I\textsubscript{Kslow} with voltage steps from -60 mV to 50 mV by 10 mV steps.} 
    \label{fig10}
\end{figure}

\section*{Conclusion}
This paper has developed a self-breeding GA method to calibrate simulation models of K\textsuperscript{+} channel of mouse ventricular apex myocytes based on key statistics and raw data obtained from in-vitro voltage-clamp experiments \cite{ednie2019reduced}. In-silico simulation models allow for the investigation of underlying dynamics of observed current and to make inferences about different experimental conditions that were not yet or cannot be conducted.

Notably, ventricular myocytes consist of several K\textsubscript{v} isoforms that lead to multiple component K\textsuperscript{+} currents. These component currents collectively produce the total I\textsubscript{Ksum} that can be recorded during in-vitro experiments. Typically, the sum of potassium currents is approximated by fitting multiple exponential functions, each of which is used to estimate the component current. However, traditional exponential functions cannot adequately describe detailed kinetics.

Simulation models were calibrated by comparing their simulated current traces with those derived from the exponential decomposition of I\textsubscript{Ksum}. The simulation models enable the prediction of K\textsubscript{v} characteristics at different voltages that may not be observed in the electrophysiological experiments. In addition to K\textsubscript{v} characteristics at 50 mV, simulation models show that changes of amplitudes, time constants, and SSI with different voltages are consistent with previous studies \cite{ednie2019reduced}.

Experimental results show that the proposed optimization method effectively calibrate simulation models of K\textsubscript{v}'s with in-vitro experimental data. These simulations models allow for the prediction of component K\textsuperscript{+} currents at different experimental conditions. In the Mgat1KO group, both I\textsubscript{Kto} and I\textsubscript{Kslow} densities are significantly reduced and the rate of I\textsubscript{Kslow} inactivation is much slower, which are consistent with experimental observations.

This study couples simulation models with in-vitro experiments to investigate the effects of reduced N-glycosylation on K\textsubscript{v} activity. A new method is used to delineate and examine multiple potassium currents. The computational simulation helps demonstrate the voltage-gating mechanism and conductance kinetics, thereby gaining a better understanding of glycosylation effects on K\textsubscript{v} kinetics.

% \section*{Supporting information}
% 
% % Include only the SI item label in the paragraph heading. Use the \nameref{label} % command to cite SI items in the text.
% \paragraph*{S1 Fig.}
% \label{S1_Fig}
% {\bf Bold the title sentence.} Add descriptive text after the title of the item % (optional).
% 
% \paragraph*{S2 Fig.}
% \label{S2_Fig}
% {\bf Lorem ipsum.} Maecenas convallis mauris sit amet sem ultrices gravida. Etiam % eget sapien nibh. Sed ac ipsum eget enim egestas ullamcorper nec euismod ligula. % Curabitur fringilla pulvinar lectus consectetur pellentesque.
% 
% \paragraph*{S1 File.}
% \label{S1_File}
% {\bf Lorem ipsum.}  Maecenas convallis mauris sit amet sem ultrices gravida. Etiam % eget sapien nibh. Sed ac ipsum eget enim egestas ullamcorper nec euismod ligula. % Curabitur fringilla pulvinar lectus consectetur pellentesque.
% 
% \paragraph*{S1 Video.}
% \label{S1_Video}
% {\bf Lorem ipsum.}  Maecenas convallis mauris sit amet sem ultrices gravida. Etiam % eget sapien nibh. Sed ac ipsum eget enim egestas ullamcorper nec euismod ligula. % Curabitur fringilla pulvinar lectus consectetur pellentesque.
% 
% \paragraph*{S1 Appendix.}
% \label{S1_Appendix}
% {\bf Lorem ipsum.} Maecenas convallis mauris sit amet sem ultrices gravida. Etiam % eget sapien nibh. Sed ac ipsum eget enim egestas ullamcorper nec euismod ligula. % Curabitur fringilla pulvinar lectus consectetur pellentesque.
% 
% \paragraph*{S1 Table.}
% \label{S1_Table}
% {\bf Lorem ipsum.} Maecenas convallis mauris sit amet sem ultrices gravida. Etiam eget sapien nibh. Sed ac ipsum eget enim egestas ullamcorper nec euismod ligula. Curabitur fringilla pulvinar lectus consectetur pellentesque.

\section*{Acknowledgments}
The authors would like thank the National Science Foundation grants MCB-1856132 and MCB-1856199 for the support of this research. Any opinions, findings, or conclusions expressed in this paper are those of the authors and do not necessarily reflect the views of the sponsor.

\nolinenumbers

% Either type in your references using
% \begin{thebibliography}{}
% \bibitem{}
% Text
% \end{thebibliography}
%
% or
%
% Compile your BiBTeX database using our plos2015.bst
% style file and paste the contents of your .bbl file
% here. See http://journals.plos.org/plosone/s/latex for 
% step-by-step instructions.
% 

% \bibliography{ref}

\begin{thebibliography}{10}

\bibitem{cdc2019deaths}
Heron M.
\newblock Deaths: Leading Causes for 2017.
\newblock National Vital Statistics Reports. 2019;68(6).

\bibitem{weintraub2017dilated}
Weintraub RG, Semsarian C, Macdonald P.
\newblock Dilated cardiomyopathy.
\newblock The Lancet. 2017;390(10092):400--414.

\bibitem{lakdawala2013dilated}
Lakdawala NK, Winterfield JR, Funke BH.
\newblock Dilated cardiomyopathy.
\newblock Circulation: Arrhythmia and Electrophysiology. 2013;6(1):228--237.

\bibitem{hershberger2011update}
Hershberger RE, Siegfried JD.
\newblock Update 2011: clinical and genetic issues in familial dilated
  cardiomyopathy.
\newblock Journal of the American College of Cardiology.
  2011;57(16):1641--1649.

\bibitem{gehrmann2003cardiomyopathy}
Gehrmann J, Sohlbach K, Linnebank M, B{\"o}hles HJ, Buderus S, Kehl HG, et~al.
\newblock Cardiomyopathy in congenital disorders of glycosylation.
\newblock Cardiology in the Young. 2003;13(4):345--351.

\bibitem{footitt2009cardiomyopathy}
Footitt E, Karimova A, Burch M, Yayeh T, Dupre T, Vuillaumier-Barrot S, et~al.
\newblock Cardiomyopathy in the congenital disorders of glycosylation (CDG): a
  case of late presentation and literature review.
\newblock Journal of Inherited Metabolic Disease: Official Journal of the
  Society for the Study of Inborn Errors of Metabolism. 2009;32:313--319.

\bibitem{marques2017cardiac}
Marques-da Silva D, Francisco R, Webster D, dos Reis~Ferreira V, Jaeken J,
  Pulinilkunnil T.
\newblock Cardiac complications of congenital disorders of glycosylation (CDG):
  a systematic review of the literature.
\newblock Journal of inherited metabolic disease. 2017;40(5):657--672.

\bibitem{ednie2019reduced}
Ednie AR, Parrish AR, Sonner MJ, Bennett ES.
\newblock Reduced hybrid/complex N-glycosylation disrupts cardiac electrical
  signaling and calcium handling in a model of dilated cardiomyopathy.
\newblock Journal of molecular and cellular cardiology. 2019;132:13--23.

\bibitem{ednie2019reduced2}
Ednie AR, Deng W, Yip KP, Bennett ES.
\newblock Reduced myocyte complex N-glycosylation causes dilated
  cardiomyopathy.
\newblock The FASEB Journal. 2019;33(1):1248--1261.

\bibitem{koenig2011voltage}
Koenig X, Dysek S, Kimbacher S, Mike AK, Cervenka R, Lukacs P, et~al.
\newblock Voltage-gated ion channel dysfunction precedes cardiomyopathy
  development in the dystrophic heart.
\newblock PLoS One. 2011;6(5):e20300.

\bibitem{grant2009cardiac}
Grant AO.
\newblock Cardiac ion channels.
\newblock Circulation: Arrhythmia and Electrophysiology. 2009;2(2):185--194.

\bibitem{bondarenko2004computer}
Bondarenko VE, Szigeti GP, Bett GC, Kim SJ, Rasmusson RL.
\newblock Computer model of action potential of mouse ventricular myocytes.
\newblock American Journal of Physiology-Heart and Circulatory Physiology.
  2004;287(3):H1378--H1403.

\bibitem{ednie2011modulation}
Ednie AR, Bennett ES.
\newblock Modulation of voltage-gated ion channels by sialylation.
\newblock Comprehensive Physiology. 2011;2(2):1269--1301.

\bibitem{ednie2013expression}
Ednie AR, Horton KK, Wu J, Bennett ES.
\newblock Expression of the sialyltransferase, ST3Gal4, impacts cardiac
  voltage-gated sodium channel activity, refractory period and ventricular
  conduction.
\newblock Journal of molecular and cellular cardiology. 2013;59:117--127.

\bibitem{ednie2015reduced}
Ednie AR, Bennett ES.
\newblock Reduced sialylation impacts ventricular repolarization by modulating
  specific K+ channel isoforms distinctly.
\newblock Journal of Biological Chemistry. 2015;290(5):2769--2783.

\bibitem{du2017}
{Du} D, {Yang} H, {Ednie} AR, {Bennett} ES.
\newblock In-Silico Modeling of the Functional Role of Reduced Sialylation in
  Sodium and Potassium Channel Gating of Mouse Ventricular Myocytes.
\newblock IEEE Journal of Biomedical and Health Informatics.
  2018;22(2):631--639.

\bibitem{jerng2021light}
Jerng HH, Patel JM, Khan TA, Arenkiel BR, Pfaffinger PJ.
\newblock Light-regulated voltage-gated potassium channels for acute
  interrogation of channel function in neurons and behavior.
\newblock Plos one. 2021;16(3):e0248688.

\bibitem{brunet2004heterogeneous}
Brunet S, Aimond F, Li H, Guo W, Eldstrom J, Fedida D, et~al.
\newblock Heterogeneous expression of repolarizing, voltage-gated K+ currents
  in adult mouse ventricles.
\newblock The Journal of physiology. 2004;559(1):103--120.

\bibitem{molina2013differential}
Molina-Navarro MM, Rosell{\'o}-Llet{\'\i} E, Ortega A, Taraz{\'o}n E, Otero M,
  Mart{\'\i}nez-Dolz L, et~al.
\newblock Differential gene expression of cardiac ion channels in human dilated
  cardiomyopathy.
\newblock PloS one. 2013;8(12):e79792.

\bibitem{barrans2002global}
Barrans JD, Allen PD, Stamatiou D, Dzau VJ, Liew CC.
\newblock Global gene expression profiling of end-stage dilated cardiomyopathy
  using a human cardiovascular-based cDNA microarray.
\newblock The American journal of pathology. 2002;160(6):2035--2043.

\bibitem{hwang2002microarray}
Hwang JJ, Allen PD, Tseng GC, Lam CW, Fananapazir L, Dzau VJ, et~al.
\newblock Microarray gene expression profiles in dilated and hypertrophic
  cardiomyopathic end-stage heart failure.
\newblock Physiological genomics. 2002;10(1):31--44.

\bibitem{yung2004gene}
Yung CK, Halperin VL, Tomaselli GF, Winslow RL.
\newblock Gene expression profiles in end-stage human idiopathic dilated
  cardiomyopathy: altered expression of apoptotic and cytoskeletal genes.
\newblock Genomics. 2004;83(2):281--297.

\bibitem{nishio1995identification}
Nishio Y, Warren CE, Buczek-Thomas JA, Rulfs J, Koya D, Aiello LP, et~al.
\newblock Identification and characterization of a gene regulating enzymatic
  glycosylation which is induced by diabetes and hyperglycemia specifically in
  rat cardiac tissue.
\newblock The Journal of clinical investigation. 1995;96(4):1759--1767.

\bibitem{knezevic2009variability}
Knezevic A, Polasek O, Gornik O, Rudan I, Campbell H, Hayward C, et~al.
\newblock Variability, heritability and environmental determinants of human
  plasma N-glycome.
\newblock Journal of proteome research. 2009;8(2):694--701.

\bibitem{miura2016glycomics}
Miura Y, Endo T.
\newblock Glycomics and glycoproteomics focused on aging and age-related
  diseases—glycans as a potential biomarker for physiological alterations.
\newblock Biochimica et Biophysica Acta (BBA)-General Subjects.
  2016;1860(8):1608--1614.

\bibitem{nagai2016aberrant}
Nagai-Okatani C, Minamino N.
\newblock Aberrant glycosylation in the left ventricle and plasma of rats with
  cardiac hypertrophy and heart failure.
\newblock PLoS One. 2016;11(6):e0150210.

\bibitem{yang2015glycoproteins}
Yang S, Chen L, Sun S, Shah P, Yang W, Zhang B, et~al.
\newblock Glycoproteins identified from heart failure and treatment models.
\newblock Proteomics. 2015;15(2-3):567--579.

\bibitem{hodgkin1952quantitative}
Hodgkin AL, Huxley AF.
\newblock A quantitative description of membrane current and its application to
  conduction and excitation in nerve.
\newblock The Journal of physiology. 1952;117(4):500.

\bibitem{mullins2013mathematical}
Mullins PD, Bondarenko VE.
\newblock A mathematical model of the mouse ventricular myocyte contraction.
\newblock PLoS One. 2013;8(5):e63141.

\bibitem{du2015statistical}
Du D, Yang H, Ednie AR, Bennett ES.
\newblock Statistical metamodeling and sequential design of computer
  experiments to model Glyco-altered gating of sodium channels in cardiac
  myocytes.
\newblock IEEE journal of biomedical and health informatics.
  2015;20(5):1439--1452.

\bibitem{plumlee2016calibrating}
Plumlee M, Joseph VR, Yang H.
\newblock Calibrating functional parameters in the ion channel models of
  cardiac cells.
\newblock Journal of the American Statistical Association.
  2016;111(514):500--509.

\bibitem{nerbonne2005molecular}
Nerbonne JM, Kass RS.
\newblock Molecular physiology of cardiac repolarization.
\newblock Physiological reviews. 2005;85(4):1205--1253.

\bibitem{schwetz2010n}
Schwetz TA, Norring SA, Bennett ES.
\newblock N-glycans modulate Kv1. 5 gating but have no effect on Kv1. 4 gating.
\newblock Biochimica et Biophysica Acta (BBA)-Biomembranes.
  2010;1798(3):367--375.

\end{thebibliography}

\end{document}
